\documentclass[12pt]{article}
\usepackage[margin=0.75in]{geometry}
\usepackage{graphicx}
\setlength{\parindent}{0mm}

\begin{document}

{\centering
\large University of North Georgia \par
\large College of Science and Mathematics \par
\large Department of Physics \par
\large PHYS 2211L - Principles of Physics I, Laboratory - Fall 2018 \par
}
\hfill \break \vspace{-4mm}

\underline{\textbf{General Information}} \par
Instructor: Dr. Nathan Harrison \par
The information contained in this lab syllabus is in addition to the lecture syllabus. Please be familiar with both syllabi.
\hfill \break

\underline{\textbf{Course Description}} \par
This course is a co-requisite of PHYS 2211 and entails laboratory investigation of the fundamental principles of mechanics.
An emphasis will be placed on introductory scientific computing and experimental data analysis.
\hfill \break

\underline{\textbf{Means by Which Outcomes will be Assessed}} \par
\begin{enumerate}
\item Students will do calculations to predict the outcomes of experiments
\item Students will perform measurements/simulations to measure the outcomes of experiments, and compare with theoretical predictions
\item Students will write laboratory reports and compile a portfolio
\item Students will take a lab final exam
\end{enumerate}

\underline{\textbf{Course Content}} \par
Each chapter covered in lecture will concluded with a laboratory investigation of relevant topics (see lecture syllabus for more details).
A laboratory final exam will take place during finals week.
\hfill \break

\underline{\textbf{Course Policies}} \par
\begin{enumerate}
\item Attendance: regular and punctual attendance is expected of all students. Two unjustified absences qualify a student for a WF grade.
\item Make-ups: make-ups will not be allowed in general. Email to inform the instructor if you are going to miss a test or a lab due to an extenuating circumstance.
\end{enumerate}

\underline{\textbf{Course Grading}} \par
Each student is expected to write a laboratory report at the end of every lab.
The report is due one week after the completion of the lab.
A late penalty of up to a letter grade per day will be incurred for late lab reports.
The course grade will be determined as follows: \par
\hfill \break
Laboratory Reports: 70\% \par
Lab Final Exam: 20\% \par
Participation: 10\%
\hfill \break

\underline{\textbf{``Full'' Laboratory Report Format}} \par
All reports must adhere to the following format with clearly labeled sections:
\begin{enumerate}
\item OBJECTIVE: A clear statement of the goals of the investigation/experiment
\item THEORY: A clear exposition of the theoretical background of the investigation. All the formulae used and a description of the variables must be included in this section at a minimum.
\item PROCEDURE: Break up the procedure followed in the experiment to a series of numbered steps
\item DATA: The original data must be attached to the report. The DATA-table must be neat, clearly labeled and with appropriate units
\item CALCULATIONS: All calculations are to be clearly shown in original. Do not type your calculations. Make sure that all the calculated quantities have the appropriate units
\item RESULTS: Summarize all the numerical results of the experiment, preferably in a table.
\item ANALYSIS: Ascertain reasons for inaccurate results if that was the case
\item COMMENTS: State any comments you may have about the experiment performed
\end{enumerate}

\underline{\textbf{``Abbreviated'' Laboratory Report Format}} \par
\begin{enumerate}
\item OBJECTIVE
\item DATA
\item CALCULATIONS
\item RESULTS
\end{enumerate}

\underline{\textbf{Lab Report Expectations}} \par
\begin{enumerate}
\item Neat
\item Organized
\item With clearly labeled data tables with appropriate units
\item All original data has to be presented
\item All calculated quantities must have units on them
\item There must be a clear RESULTS section summarizing the numerical results of the experiment; this should preferably be in tabular form
\end{enumerate}
All the lab reports are expected to be in a binder/folder, such that the student will compile a Laboratory Portfolio in the course of the semester.
The entire portfolio will be due at the time of the lab final exam.
\hfill \break

\underline{\textbf{Grading Criteria}} \par
\begin{enumerate}
\item Neat presentation of write-up (minus 4 if not)
\item Organized in a binder with labeled dividers for each lab (minus 5 if not)
\item With clearly labeled data tables with appropriate units (minus 8 if not)
\item All original data has to be presented (minus 5 if not)
\item All calculated quantities must have units on them (minus 4 for each instance)
\item There must be a clear RESULTS section summarizing the numerical results (with appropriate units) of the experiment; this must be in a tabular form (minus 8 if no Results section)
\item If a section is missing – minus 15
\item Misrepresented result (reported results not asked for) – minus 5 per case
\item Missing error analysis – minus 5 per case
\item Incorrect Calculation – minus 8 per case
\item Incomplete Theory (missing formulae, description of variables etc) - minus 8
\item Attention to significant figures – minus 2 for each case
\end{enumerate}


\end{document}
