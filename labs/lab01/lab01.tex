\documentclass[12pt]{article}
\usepackage[margin=0.75in]{geometry}
\usepackage{graphicx}
\setlength{\parindent}{0mm}

\begin{document}

{\centering
\large Physics 1111: Lab 01 \par
\large Surface area, volume, and density of various objects \par
}
\hfill \break \vspace{-4mm}

\underline{\textbf{Instructions}} \par
You will be studying the properties of 3 objects - a rectangular prism, a sphere, and a (partially) hollow cylinder.
Don’t forget to use the appropriate number of significant figures, units, and error bars.
When taking a measurement, use half of the smallest reading on the instrument as the error bar;
for example, for a ruler with mm markings, the error bar should be 0.5 mm.
Use the ``$\pm$'' symbol to report the error bar; for example, Width = $3.55 \pm 0.05$ cm.
\begin{enumerate}
\item Determine the mass of each object by using the electronic balance in the front of the room and record the results.
\item Determine the relevant dimensions of each object, measure them with the calipers, and record the results. 
\item Use your measurements to calculate the surface area, volume, and density of each object. Remember to propagate the error and report the error bar.
\item Use available resources to identify the material of the object. Site your source(s).
\end{enumerate}

\end{document}
